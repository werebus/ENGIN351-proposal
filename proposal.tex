\documentclass{beamer}

\usetheme{Frankfurt}
\usepackage{lmodern}
\usepackage{localcolorscheme}
\setbeamertemplate{items}[default]
\setbeamertemplate{sections/subsections in toc}[default]

\usepackage{hyperref}
\definecolor{links}{HTML}{2A1B81}
\hypersetup{colorlinks,linkcolor=,urlcolor=links}

\title{Proposal Project: Engineering 351}
\subtitle{RFP No. 2014-04}
\author{Matt Moretti \\
\href{mailto:moretti@admin.umass.edu}%
            {moretti@admin.umass.edu}}
\institute{ENGIN 351 \\
           University of Massachusetts}
\date{April 30, 2014}

%\AtBeginSection[]
%{
  %\begin{frame}
    %\frametitle{Table of Contents}
    %\tableofcontents[currentsection]
  %\end{frame}
%}

\newcommand{\ema}[2]{#1~$\langle\text{#2}\rangle$}

\begin{document}
\frame{\titlepage}

\section*{Outline}
\begin{frame}
  \frametitle{Table of Contents}
  \tableofcontents
\end{frame}

\section{Introduction}
\begin{frame}
  \frametitle{Introduction}
  This proposal is in to response to request number 2014-04, a request for
  engineering students to perform as-needed technical writing services.  In
  the following slides, I will lay out my educational and professional
  background and present the capabilities and experience that I feel make me
  a good choice for this project.
\end{frame}

\subsection{Educational Background}
\begin{frame}
  \frametitle{Educational Background}

  I have a strong background in both transportation engineering and computer
  science.

  \begin{itemize}
    \item B.S. Civil and Environmental Engineering \\
      Concentration in Transportation Engineering
    \item Minor in Computer Science
    \item Certificate in Transit Operations Management
  \end{itemize}
\end{frame}

\subsection{Professional Background}
\begin{frame}
  \frametitle{Professional Background}

  I am currently employed with UMass Transit Services as an IT Systems Engineer,
  and have conducted many professional correspondences in that capacity.

  \begin{itemize}
    \item Worked in IT for the last seven years
    \item Worked in transit management for the last ten years
    \item Been involved in the daily operation of the transit system for the
      last thirteen years
  \end{itemize}
\end{frame}

\section{Qualifications}
\begin{frame}
  \frametitle{Qualifications}
  I have a number of qualifications that will be required over the course of
  this project. They include\ldots
\end{frame}

\subsection{Professional Correspondence}
\begin{frame}
  \frametitle{Proficient With a Variety of Professional Correspondence}
  In the course of my professional and educational career, I have become
  proficient with a variety of forms of professional correspondence,
  including, but mot limited to:
  \begin{itemize}
    \item Summaries
    \item Technical Memos
    \item Emails
    \item Posters
    \item Process Descriptions
    \item Instruction Sets
  \end{itemize}
\end{frame}

\subsection{Data Awareness}
\begin{frame}
  \frametitle{Conscious of the Importance of Clear and Accurate Data}
  As an engineer, I'm very conscious of the need to be able to present data in
  a way that is clear, unambiguous, and honest.  I have experience both
  writing in such a way about numbers, and presenting those numbers in tables,
  charts, and graphs.
\end{frame}

\subsection{Technical Skills}
\begin{frame}
  \frametitle{A Variety of Technical Skills}
  Owing largely to my career in IT which has included among its duties
  creating and maintaining technical documentation and source code, I am
  familiar with a variety of software and markup formats including
  \begin{itemize}
    \item Microsoft Word
    \item Libre Office
    \item git
    \item MediaWiki
    \item Markdown
    \item \LaTeX
    \item HTML and CSS
  \end{itemize}
  \begin{alertblock}{In fact\ldots}
    This presentation itself is written in \LaTeX, and the source is available
    \href{http://github.com/werebus/ENGIN351-proposal}{on Github}.
  \end{alertblock}
\end{frame}

\subsection{Rhetorical Awareness}
\begin{frame}
  \frametitle{Aware of the Importance of the Rhetorical Situation}
  My instruction in technical writing has taught me that it is important to
  consider the content, the speaker, \emph{and} the audience before beginning
  a piece of writing.  I will continue to exercise that skill in any writing
  performed for this project.
\end{frame}

\section{Sample Recent Work}
\subsection{Email}
\begin{frame}
  \frametitle{Email}
  Below is a sample email written to a colleague introducing myself and our
  content management system.

\fbox{
\begin{minipage}[t]{0.7\textwidth}
  \Tiny
  \begin{description}
    \setlength{\itemsep}{0pt}
    \setlength{\parskip}{0pt}
    \item [{To:}] \ema{Amir Khan}{akhan@globalsystems.com}
    \item [{CC:}] \ema{Leah Allen}{lallen@globalsystems.com}
    \item [{From:}] \ema{Matt Moretti}{mmoretti@globalsystems.com}
    \item [{Date:}] \today
    \item [{Subject:}] Getting started with POODLE
  \end{description}

  Hello Amir,

  My name is Matt and I'm a member of the electrical engineering team here.
  I wanted to welcome you to Global Systems, and I look forward to us working
  together soon.

  My supervisor, Leah, asked me to get in touch with you to describe the process of
  logging in to our online task management system, POODLE.  Here is a quick
  overview of the process:

  \begin{enumerate}
    \item Visit http://www.poodle.globalsystems.com in your web browser.
    \item Click on the button labeled ``Log in with NetID'' in the
      upper-left-hand corner.
    \item On the next screen, enter your user name and password in the boxes
      provided and click on the grey arrow to log in.
    \item Once you are logged in, you will be on the home page.  At the top of
      the left-hand column you will find a list of projects; click on one to go
      to the project's page.
    \item The project page is divided into three sections: The left column
      contains navigation links and settings. The right column contains the
      project news, events, and calendar. And, the main area in the center
      contains all of the project resources grouped into sub-projects.
    \item When you're done looking around, the ``Logout'' link is in the
      upper-right corner of the page.
  \end{enumerate}

  I hope this will serve to get you started, but don't hesitate to write me back
  back if you have any problems or questions.

  Matt Moretti
\end{minipage}
}
\end{frame}

\subsection{Instruction Set}
\begin{frame}
  \frametitle{Instruction Set}
  A set of instructions that I wrote on how to use the \texttt{iptables} Linux
  firewall can be found here:

  \href{https://gist.github.com/werebus/d8385ac79f0ff01dc212}%
       {https://gist.github.com/werebus/d8385ac79f0ff01dc212}
\end{frame}

\begin{frame}
  Other samples are available upon request.
\end{frame}

\section{Compensation}
\begin{frame}
  \frametitle{Compensation}
  My compensation schedule is as follows:
  \begin{center}
    \begin{tabular}[c]{|l|r|}
      \hline
      \textbf{Service} & \textbf{Price} \\ \hline
      Lay Summary & \$80 per page summarized \\
      Technical Summary & \$100 per page summarized \\
      Process Description / Instruction Set & \$100 per page  \\
      Business Correspondence & \$70 per page\\
      Consulting \& Editing & \$40 per hour \\
      Consulting \& Editing (On-site) & \$50 per hour \\
      \hline
    \end{tabular}
  \end{center}
\end{frame}
    
\section{Conclusion}
\begin{frame}
  \frametitle{Conclusion}
  I believe that I have demonstrated that I possess the skills in technical
  writing required by Lehman Industries for their upcoming project.  I hope
  you will consider selecting me as your writer.  If you have any questions or
  require additional information, please contact me.

  \begin{center}
    {\Large Thank You}
  \end{center}
\end{frame}

\end{document}
